% Template for PLoS
% Version 3.4 January 2017
%
% % % %textcolor{red} % % % % % % % % % % % % % % % % % %
%
% -- IMPORTANT NOTE
%
% This template contains comments intended 
% to minimize problems and delays during our production 
% process. Please follow the template instructions
% whenever possible.
%
% % % % % % % % % % % % % % % % % % % % % % % 
%
% Once your paper is accepted for publication, 
% PLEASE REMOVE ALL TRACKED CHANGES in this file 
% and leave only the final text of your manuscript. 
% PLOS recommends the use of latexdiff to track changes during review, as this will help to maintain a clean tex file.
% Visit https://www.ctan.org/pkg/latexdiff?lang=en for info or contact us at latex@plos.org.
%
%
% There are no restrictions on package use within the LaTeX files except that 
% no packages listed in the template may be deleted.
%
% Please do not include colors or graphics in the text.
%
% The manuscript LaTeX source should be contained within a single file (do not use \input, \externaldocument, or similar commands).
%
% % % % % % % % % % % % % % % % % % % % % % %
%
% -- FIGURES AND TABLES
%
% Please include tables/figure captions directly after the paragraph where they are first cited in the text.
%
% DO NOT INCLUDE GRAPHICS IN YOUR MANUSCRIPT
% - Figures should be uploaded separately from your manuscript file. 
% - Figures generated using LaTeX should be extracted and removed from the PDF before submission. 
% - Figures containing multiple panels/subfigures must be combined into one image file before submission.
% For figure citations, please use "Fig" instead of "Figure".
% See http://journals.plos.org/plosone/s/figures for PLOS figure guidelines.
%
% Tables should be cell-based and may not contain:
% - spacing/line breaks within cells to alter layout or alignment
% - do not nest tabular environments (no tabular environments within tabular environments)
% - no graphics or colored text (cell background color/shading OK)
% See http://journals.plos.org/plosone/s/tables for table guidelines.
%
% For tables that exceed the width of the text column, use the adjustwidth environment as illustrated in the example table in text below.
%
% % % % % % % % % % % % % % % % % % % % % % % %
%
% -- EQUATIONS, MATH SYMBOLS, SUBSCRIPTS, AND SUPERSCRIPTS
%
% IMPORTANT
% Below are a few tips to help format your equations and other special characters according to our specifications. For more tips to help reduce the possibility of formatting errors during conversion, please see our LaTeX guidelines at http://journals.plos.org/plosone/s/latex
%
% For inline equations, please be sure to include all portions of an equation in the math environment.  For example, x$^2$ is incorrect; this should be formatted as $x^2$ (or $\mathrm{x}^2$ if the romanized font is desired).
%
% Do not include text that is not math in the math environment. For example, CO2 should be written as CO\textsubscript{2} instead of CO$_2$.
%
% Please add line breaks to long display equations when possible in order to fit size of the column. 
%
% For inline equations, please do not include punctuation (commas, etc) within the math environment unless this is part of the equation.
%
% When adding superscript or subscripts outside of brackets/braces, please group using {}.  For example, change "[U(D,E,\gamma)]^2" to "{[U(D,E,\gamma)]}^2". 
%
% Do not use \cal for caligraphic font.  Instead, use \mathcal{}
%
% % % % % % % % % % % % % % % % % % % % % % % % 
%
% Please contact latex@plos.org with any questions.
%
% % % % % % % % % % % % % % % % % % % % % % % %

\documentclass[10pt,letterpaper]{article}
\usepackage[top=0.85in,left=2.75in,footskip=0.75in]{geometry}

% amsmath and amssymb packages, useful for mathematical formulas and symbols
\usepackage{amsmath,amssymb}

% Use adjustwidth environment to exceed column width (see example table in text)
\usepackage{changepage}

% Use Unicode characters when possible
\usepackage[utf8x]{inputenc}

% textcomp package and marvosym package for additional characters
\usepackage{textcomp,marvosym}

% cite package, to clean up citations in the main text. Do not remove.
\usepackage{cite}

% Use nameref to cite supporting information files (see Supporting Information section for more info)
\usepackage{nameref,hyperref}

% line numbers
\usepackage[right]{lineno}

% ligatures disabled
\usepackage{microtype}
\DisableLigatures[f]{encoding = *, family = * }

% color can be used to apply background shading to table cells only
\usepackage[table]{xcolor}

% array package and thick rules for tables
\usepackage{array}
%Ad hoc for Table H
\usepackage{float}

\usepackage{xcolor}
% create "+" rule type for thick vertical lines
\newcolumntype{+}{!{\vrule width 2pt}}

% create \thickcline for thick horizontal lines of variable length
\newlength\savedwidth
\newcommand\thickcline[1]{%
  \noalign{\global\savedwidth\arrayrulewidth\global\arrayrulewidth 2pt}%
  \cline{#1}%
  \noalign{\vskip\arrayrulewidth}%
  \noalign{\global\arrayrulewidth\savedwidth}%
}

% \thickhline command for thick horizontal lines that span the table
\newcommand\thickhline{\noalign{\global\savedwidth\arrayrulewidth\global\arrayrulewidth 2pt}%
\hline
\noalign{\global\arrayrulewidth\savedwidth}}


% Remove comment for double spacing
%\usepackage{setspace} 
%\doublespacing

% Text layout
\raggedright
\setlength{\parindent}{0.5cm}
\textwidth 5.25in 
\textheight 8.75in

% Bold the 'Figure #' in the caption and separate it from the title/caption with a period
% Captions will be left justified
\usepackage[aboveskip=1pt,labelfont=bf,labelsep=period,justification=raggedright,singlelinecheck=off]{caption}
\renewcommand{\figurename}{Fig}

% Use the PLoS provided BiBTeX style
\bibliographystyle{plos2015}

% Remove brackets from numbering in List of References
\makeatletter
\renewcommand{\@biblabel}[1]{\quad#1.}
\makeatother

% Leave date blank
\date{}

% Header and Footer with logo
\usepackage{lastpage,fancyhdr,graphicx}
\usepackage{epstopdf}
\pagestyle{myheadings}
\pagestyle{fancy}
\fancyhf{}
\setlength{\headheight}{27.023pt}
\lhead{\includegraphics[width=2.0in]{PLOS-submission.eps}}
\rfoot{\thepage/\pageref{LastPage}}
\renewcommand{\footrule}{\hrule height 2pt \vspace{2mm}}
\fancyheadoffset[L]{2.25in}
\fancyfootoffset[L]{2.25in}
\lfoot{\sf PLOS}

%% Include all macros below

\newcommand{\lorem}{{\bf LOREM}}
\newcommand{\ipsum}{{\bf IPSUM}}


\usepackage{booktabs}
\usepackage{xcolor}
\usepackage[normalem]{ulem} % strikeout (\sout{text})

%% END MACROS SECTION


\begin{document}
\vspace*{0.2in}

% Title must be 250 characters or less.
\begin{flushleft}
{\Large
\textbf\newline{Inattention in primary school is not good for your future school achievement - a pattern classification study} % Please use "sentence case" for title and headings (capitalize only the first word in a title (or heading), the first word in a subtitle (or subheading), and any proper nouns).
}
\newline
% Insert author names, affiliations and corresponding author email (do not include titles, positions, or degrees).

Astri J. Lundervold\textsuperscript{1,2*},
Tormod B\o{}e\textsuperscript{3},
Arvid Lundervold\textsuperscript{4}


\bigskip
\textbf{1} Department of Biological and Medical Psychology University of Bergen, 5009 Bergen, Norway
\\
\textbf{2} K.G. Jebsen Center for Research on Neuropsychiatric Disorders, University of Bergen, Bergen, Norway\\
\textbf{3} Regional Centre for Child and Youth Mental Health and Child Welfare, Uni Research Health, Bergen, Norway\\
\textbf{4} Neuroinformatics and Image Analysis Laboratory, Neural Networks Research Group, Department of Biomedicine, University of Bergen
\\
\bigskip

% Insert additional author notes using the symbols described below. Insert symbol callouts after author names as necessary.
% 
% Remove or comment out the author notes below if they aren't used.
%
% Primary Equal Contribution Note
%\Yinyang These authors contributed equally to this work.

% Additional Equal Contribution Note
% Also use this double-dagger symbol for special authorship notes, such as senior authorship.
%\ddag These authors also contributed equally to this work.

% Current address notes
%\textcurrency Current Address: Dept/Program/Center, Institution Name, City, State, Country % change symbol to "\textcurrency a" if more than one current address note
% \textcurrency b Insert second current address 
% \textcurrency c Insert third current address

% Deceased author note
%\dag Deceased

% Group/Consortium Author Note
%\textpilcrow Membership list can be found in the Acknowledgments section.

% Use the asterisk to denote corresponding authorship and provide email address in note below.
* Corresponding author: astri.lundervold@uib.no

\end{flushleft}
% Please keep the abstract below 300 words
\section*{Abstract}
\noindent \emph{Objective.}
Inattention in childhood is associated with academic problems later in life. The contribution of specific aspects of inattentive behaviour is, however, less known. We investigated feature importance of primary school teachers' reports on nine aspects of inattentive behaviour, gender and age in predicting future academic achievement. 

\noindent \emph{Methods.} Primary school teachers of $n=2491$ children (7 - 9 years) rated nine items reflecting different aspects of inattentive behaviour in 2002. \textcolor{red}{A mean academic achievement score at high-school (2012) was available for each youth from an official school register. All scores were at a categorical level.} \textcolor{blue}{Feature importances were assessed by using  classification and regression trees analysis, multinominal logistic regression, and a random forest algorithm}. Finally, a comprehensive pattern classification procedure using $k$-fold cross-validation was implemented.

\noindent \emph{Results.}
Overall, inattention was rated as more severe in boys, who also obtained lower \textcolor{blue}{academic} achievement scores at high school than girls. Problems related to sustained attention and distractibility were together with age and gender defined as the most important features to predict future achievement scores. 
\textcolor{red}{\sout{A cross-validation model including these four features gave accuracy}, \sout{precision and recall that were above chance levels.} } 
\textcolor{blue}{Using these four features as input to the collection of  classifiers
employing $k$-fold cross-validation for prediction of academic achievement level, we obtained classification accuracy, precision and recall that were clearly better than chance levels. }

\noindent \emph{Conclusion} 
Primary school teachers' reports of problems related to sustained attention and distractibility were identified as the \textcolor{blue}{two} most important features of inattentive behaviour predicting academic achievement in high school.  Identification and follow-up procedures of primary school children showing these \textcolor{blue}{characteristics} \sout{features} should be prioritised to prevent future academic failure. %}


% Please keep the Author Summary between 150 and 200 words
% Use first person. PLOS ONE authors please skip this step. 
% Author Summary not valid for PLOS ONE submissions.   
%\section*{Author summary}
%Lorem ipsum dolor sit amet, consectetur adipiscing elit. Curabitur eget porta erat. Morbi consectetur est vel gravida pretium. Suspendisse ut dui eu ante cursus gravida non sed sem. Nullam sapien tellus, commodo id velit id, eleifend volutpat quam. Phasellus mauris velit, dapibus finibus elementum vel, pulvinar non tellus. Nunc pellentesque pretium diam, quis maximus dolor faucibus id. Nunc convallis sodales ante, ut ullamcorper est egestas vitae. Nam sit amet enim ultrices, ultrices elit pulvinar, volutpat risus.

\linenumbers

% Use "Eq" instead of "Equation" for equation citations.
\section*{Introduction}
Inattention in early childhood has been linked to a wide range of behavioural and social problems \cite{Bellanti2000, Connors2012},  including poor academic achievement. This has been shown in several studies of individuals with Attention Deficit Hyperactivity disorder 
(ADHD, see e.g. \cite{Lee2006, Owens2017}), but is also widely documented in studies including community samples \cite{Polderman2010, Metcalfe2013, Pingault2014, Garner2013, Holmberg2014, Gray2014, Salla2016}.  In most of these studies, inattention is defined as a sum score across a set of items.

Inattention is, however, a multidimensional concept,  where the items 
reflect impairment of sustained and focused attention, impaired working memory, distractibility, forgetfulness, as well as impaired ability to organise and plan activities and tasks. These aspects of inattention have been described as independent at a biological level  \cite{Berry2014}, but may be extremely difficult to disentangle behaviourally. They rather tend to occur as patterns of behaviour. For example, most children may be distracted by external stimuli in a classroom situation \cite{Rescorla2007}, and these distractions will probably be especially hard to handle by a child who has problems maintaining attention and engagement in a task. Thus, it may not be the total inattention score, but rather specific patterns of inattentive behaviour that have the most detrimental effect on the child's present and future function at school. Identification of important features of inattentive behaviour will therefore be of great importance when developing remediation procedures. 

Primary school teachers' skills are crucial in the work to detect and help a child struggling with inattention. They observe their pupils on a regular basis and in a wide range of situations were inattention tends to have negative effects on performance. At the same time, one should be aware of the risk of biases. Primary school teachers may for example be more tolerant to the behaviour of a child in the lowest class levels, and previous studies have shown that teachers tend to rate girls as less impaired than boys, even when the girls exhibit problematic behaviour in the classroom \cite{Becker2013, Bussing2003c, Ohan2009}. The child's gender and age should therefore be taken into account when evaluating teacher ratings of inattentive behaviour. \\

The aim of the present study was to further investigate the importance of primary school teachers reports of inattentive behaviour. To that end, we included data from the Bergen Child Study, where primary school teachers completed a questionnaire including nine items reflecting different aspects of inattentive behaviour when the children were between 7 and 9 years old. About ten years later, when the children had become high school students, academic achievement scores from the official school registry of Norway were available for a subset of the children from the original sample. \textcolor{red}{Described as a key determinant of later vocational career success and adult
financial stability \cite{Fried2016}, there are strong arguments for using academic achievement as an outcome variabel.} Each of the nine inattention items were rated on a Likert secale with three response alternatives, and the outcome variable, academic achievement, was discretised into three intervals, including an almost equal number of participants in each category.  Teacher scores on each of the nine items were used as predictors together with gender and primary school class level (a proxy for age) to answer the following questions: ({\bf1}) which features of inattentive behaviour in primary school represent the strongest predictors of academic achievement in high school? ({\bf 2}) \textcolor{red}{how well can the result be generalised to an independent data set?}, and ({\bf3}) are gender and the age of the child when evaluated by their primary school teachers of importance to the prediction?  \\ 

In this context, statistical \textcolor{blue}{machine learning} approaches were selected according to the following criteria: ({\bf i}) the methods must handle multiple predictors with a small set of response alternatives, and with a small set of outcome categories; ({\bf ii}) the methods should be generic and of interest to other similar data analysis situations  
and prediction challenges  occurring in the behavioural sciences, and ({\bf iii}) the methods should produce results that are easy to interpret at a clinical level. 
Based on these criteria we selected \sout{a} {\em multinomial logistic regression} (MLR), \sout{a} {\em classification and regression trees} (CART), \textcolor{red}{and a {\em random forest algorithm} (RF) to assess feature importance, and a {\em k-fold cross-validation procedure} to estimate the \textcolor{blue}{classification accuracy,  precision, and recall of a model using, in the prediction, the most important features being identified, including gender and age at assessment.} \sout{the features identified as most important}. }

%The CART is a multivariate analysis method belonging to the class of recursive partitioning algorithms. 
%These methods have recently become a valuable tool for 
%exploring complex datasets in psychology where the algorithms generate decision trees that aims to correctly classify members of a sample. This is obtained by splitting 
%them into subsamples based on several binary decisions operating hierarchically on the values of the independent predictor variables. These decisions
%are easy to interpret and visualise. Together they reveal class predictive patterns of the independent variables in the sample, such that the decision tree can generalise 
%and be used to classify new cases. Being trained and used for such predictive purposes, the methods are well established in the field of machine learning. 
%

\section*{Materials and methods}
\textcolor{red}{The data included in the present study are from the Bergen Child Study (BCS), a longitudinal, population-based study on mental health and development}
The first wave of the BCS was launched in October 2002, and included the total population of 9,430 children attending second to fourth grade (7-9 years old, born in 1993, 1994 and 1995) in all public, private, and special schools in Bergen. 
During the initial screening phase, parents and teachers were asked to complete a four-page questionnaire, including, among other scales, a somewhat modified Swanson, Nolan, and Pelham Questionnaire - Fourth Edition (SNAP-IV) \cite{Swanson1992}.
Sample protocols of the first wave have been described in several previous publications from the Bergen Child Study group 
(e.g., \cite{Heiervang2007, Lundervold2011, Sivertsen2015}).\\

A fourth and \textcolor{red} {final} study-wave was conducted when the youth were between 16 and 19 years old. The sample for this wave included all adolescents born between 1993 and 1995 living n the county of Hordaland $n=10,222$). This county includes the city of Bergen, and the BCS sample was thus nested within this Hordaland sample. \textcolor{red} { Academic achievement scores  during the previous semester were made available from the official school registry.}  The BCS was approved by the Regional Committee for Medical and Health Research Ethics (REC), Western Norway (2015/800 Barn i Bergen/ung$@$hordaland). Parents gave written consent for participation in the first wave of the study. In accordance with the regulations from the REC and Norwegian health authorities, adolescents aged 16 years and older can make decisions regarding their own health (including participation in health studies), and thus gave consent themselves to participate in the fourth wave of the study. Parents/guardians have the right to be informed, and in the current study, all parents/guardians received written information about the study in advance.
\textcolor{red}{More information about the project is given at the BCS homepage: {\small \url{http://uni.no/en/bergen-child-study}}.}

\vspace{3mm}
\subsection*{The sample}
The sample selected for the present study included the participants with teacher reports on all selected SNAP-IV items when they were $7$ to $9$ years old (primary school class levels $2$, $3$, or $4$), information about gender, and academic achievement when they attended high school ($16$ to $19$ years old), \textcolor{red}{i.e., $n=2491$ participants ($1192$ boys)}
\textcolor{red}{Within this sample, the percentages of children attending 2$^{\text{nd}}$, 3$^{\text{rd}}$ and 4$^{\text{th}}$ primary school
 class levels when evaluated by their teachers were 
\textcolor{red}{42.3\%, 34.4\% and 23.4\%}, respectively.}

\vspace{3mm}
\subsection*{Teacher reports}
\emph{Inattention} items were selected from the SNAP-IV \cite{Swanson1992}, describing problems used to define the inattentive symptoms of the Attention Deficit Hyperactivity Disorder (ADHD) according to the Diagnostic and Statistical Manual of Mental Disorders (DSM-5) \cite{APA2013}. The original SNAP-IV uses four levels to evaluate each item, whereas in our study, the teachers evaluated each item on a 3-level Likert-type scale (\textcolor{blue}{``not true", ``somewhat true", or ``certainly true"}) in order to follow the response pattern of the remaining scales included in the first wave of the BCS questionnaire. Each answer was assigned a value $0$, $1$, or $2$. 
The  nine inattention items from SNAP-IV are listed in Table~1. 


\begin{table}[!ht]
\begin{adjustwidth}{-5mm}{0in} % Comment out/remove adjustwidth environment if table fits in text column.
\centering
\caption{\bf SNAP items, scored as ``not true" (0), ``somewhat true" (1), and ``certainly true" (2)).}
\vspace{5mm}
\begin{tabular}{|ll|}
\hline
{\it SNAP1} : & Often fails to give close attention to details or makes careless\\ & mistakes in schoolwork, work, or other activities\\ \hline
{\it SNAP2} : & Often has difficulty sustaining attention in tasks or play activities \\ \hline
{\it SNAP3} : & Often does not seem to listen when spoken to directly \\ \hline
{\it SNAP4} : & Often does not follow through on instructions and fails to finish\\ &  schoolwork, chores, or duties\\ \hline
{\it SNAP5} : & Often has difficulty organizing tasks and activities \\ \hline
{\it SNAP6} : & Often avoids, dislikes, or is reluctant to engage in tasks that require\\ & sustained mental effort\\ \hline
{\it SNAP7} : & Often loses things necessary for tasks or activities (e.g., toys, school\\ &  assignments, pencils, books, or tools) \\ \hline
{\it SNAP8} : & Often is distracted by extraneous stimuli\\ \hline
{\it SNAP9} : & Often is forgetful in daily activities \\ \hline
\end{tabular}
\label{Table1}
\end{adjustwidth}
\end{table}

%\vspace{-4mm}


The percentages of children scored within the three response categories are given in Table 2, confirming that the  frequency of girls reported with a (``not true") response was significantly higher than in boys. \\

\vspace{5mm}

\begin{table}[!ht]
\begin{adjustwidth}{-5mm}{0in} % Comment out/remove adjustwidth environment if table fits in text column.
\centering
\caption{ \bf Percentage of children obtaining a given response from their teachers on each inattention item (SNAP-IV).} 
\vspace{5mm}
\begin{tabular}{|lrrr|rrr|rrr|}
\hline
                & \multicolumn{3}{c|}{``Not true"}  &  \multicolumn{3}{c}{``Somewhat true"}\ &  \multicolumn{3}{|c|}{``Certainly true"}\\ \hline
                    & All  (\%) & Girls & Boys  & All (\%)  & Girls & Boys & All  (\%) & Girls & Boys\\  
                     $n$=&2167 & 1186 & 981 &278 &97 & 181 &46 & 16 & 30\\\hline
{\it SNAP1}   & 87.0 & 91.3 & 82.3**  & 11.2 & 7.5   & 15.2  & 1.8 & 1.2 & 2.5\\  \hline
{\it SNAP2}  &88.5 & 94.1   & 82.5** & 9.5 & 5.5 & 13.8 & 2.0 & 0.5 & 3.7\\  \hline
{\it SNAP3}   & 92.0  &  96.7 &  86.9** & 7.4 & 3.1 & 12.1 & 0.6 & 0.2 & 1.0 \\   \hline
{\it SNAP4}   & 92.5  & 96.2   & 88.4**  & 6.8 & 3.6 & 10.4 & 0.7& 0.2 & 1.1 \\   \hline
{\it SNAP5}  &  91.5 & 96.1 & 86.5**  & 7.3 & 3.5  & 11.4  &1.2 & 0.5 & 2.1  \\   \hline
{\it SNAP6}   & 91.7 & 96.3 & 86.7**  & 7.0 & 3.3 & 11.1 & 1.3 & 0.4 & 2.3 \\   \hline
{\it SNAP7}   & 96.5 & 98.5   & 94.4**   & 3.0  & 1.2 & 4.9 & 0.5 & 0.3 & 0.7 \\   \hline
{\it SNAP8}   & 75.2 & 84.4   & 65.1**  & 21.0  & 14.1  & 28.4 &  3.9 & 1.5 & 6.5\\    \hline
{\it SNAP9}   & 89.5 & 93.4 & 85.2** & 9.3 & 6.2 & 12.7 & 1.2 & 0.4  & 2.1 \\  \hline
\end{tabular}
\label{Numerical_SNAP}
\end{adjustwidth}
\textit{Note:} {Total number of children = $2491$, girls = $1299$, boys = $1192$.  **: $p$ value $<$.001 according to a chi-square test comparing a ``not true'' report in boys and girls.} \\ 
\end{table}



\subsection*{Academic achievement}
Academic achievement scores were provided by the official registers from the Hordaland County. In Norway, secondary schools use a scale spanning from $1$ to $6$, with $6$ 
being the highest grade (outstanding competence),  $2$  the lowest passing grade (low level of competence), and $1$ being a {\it fail}. 
\textcolor{red}{The scores included in the present study were the mean value of the grades during the previous semester, comprising all school subjects except for physical education. }
The mean score for girls was statistically significant higher  \textcolor{red}{($\mu$ = $4.11$ (SD = $0.72$)) than for boys {($\mu$ = $3.90$ (SD = $0.72$)}}, $p  <  .001$). 
For the present study, the academic achievement scores were categorised into three levels, calculated to generate groups with similar number of participants (see details below). 



\subsection*{Statistical analysis}
The data analysis was divided into \textcolor{red}{three parts: ({\bf  a}) data preparation, including discretising the average academic achievement into three levels  ({\bf b}) casting the data analysis problem 
into a machine learning classification task assessing feature importance using both 
a classification and regression tree (CART), multinomial logistic regression (MLR), and a random forest (RF) algorithm, and ({\bf c}) a pattern 
classification procedure using $k$-fold cross-validation with five different linear and nonlinear classifiers (MLR, MLP, XGB, SVM, KNN, each described below) and incorporation of
a voting classifier across  these five.
All these steps were implemented in {\tt Jupyter notebooks}  using Python (3.5.4), Numpy (1.12), Pandas (0.20), Statsmodels (0.8),  XGboost (0.6), Scikit-learn (0.19),  rpy2 (2.8.5)
and Matplotlib (2.0) for producing Figs.~1-3. Our Jupyter notebook \textcolor{blue}{for computing feature importances and 
classification with $k$-fold cross-validation} will be available on GitHub [address TBA].}  

\vspace{3mm}
\subsubsection*{(a) Data preparation and explorative data analysis}

The original data, provided to us as a SPSS-file, were imported into \textcolor{red}{the Jupyter notebook environment via {\tt rpy2} and the {\tt r-foreign} packages.} 
For the analysis we used the sample of  $n=2491$ children having complete data 
on the $11$ predictor variables and mean academic achievement as outcome variable, \textcolor{blue}{cfr. Fig.~1}.

 For classification purposes, the average academic achievement scores (\emph{ave}) were discretised into three intervals (level of academic achievement) 
 \textcolor{red}{using Pandas {\tt qcut()}, to include about the same number of participants in each of the categories:  
{\it low} (${\text{\it ave}} \in [1.000 - 3.714\rangle$,  $n$ = 834),
{\it medium} (${\text{\it ave}} \in [3.714 - 4.375\rangle$, $n$ = 831), and
{\it high} (${\text{\it ave}} \in [4.375 - 6.000]$, $n$ = 826). }
The distribution across the three levels - from {\it low} to {\it high} - was 
\textcolor{red}{40.3\%, 33.3\% and 26.4\% for boys, and 27.3\%, 33.4\% and 39.3\% for girls}, 
confirming the overall higher academic scores achieved by the girls. 

\textcolor{red}{Depiction of the complete dataset is given in Fig.~1,  using gray scale heatmap columns for the $n=2491$}  \textcolor{blue}{participants}  comprising 
the predictor variables {\it gender}, {\it grade}, 
{\it SNAP1}, ..., {\it SNAP9}, and the outcome variable {\it academic achievement}. In Fig.~1 we have also listed the
 six classifiers being used for prediction in a $k$-fold cross validation scheme. \textcolor{blue}{The observations above the horizontal dotted line represent girls and below the dotted line are the boys.}
 % .d the classification \colorbox{yellow}{and cross-validation} methods being used are given in Fig.~1 
 % Fig.~\ref{Data_to_classes}.
 
 \vspace{3mm}
 
 \begin{center}
 ---------------------------------\\
 
 \textcolor{blue}{Figure 1 about here} \\
 
  ---------------------------------\\
  \end{center}
  
  \vspace{3mm}
  
 
 \subsubsection*{(b) Assessment of feature importance} 
  \textcolor{red}{To assess feature importances of the 11 candidate variables for predicting {\it low}, {\it medium}, and {\it high} academic achievement in the whole cohort,
 we performed three types of analyses: ({\bf i}) Multinomial logistic regression with consideration of each parameter, i.e. the magnitude of its coefficient, the standard error of the corresponding parameter, and the odds ratio (Tab.~3). 
 ({\bf ii}) A CART analysis with assessment 
 of the top important decision nodes (Fig.~2), and ({\bf iii}) a random forest classification using a forest of 10000 trees (``weak learners") and ordering of features importance according
  to the 'gini' information criterion (Fig.~3). }
 


% \subsubsection*{Multinomial logistic regression model (MLR)}
\paragraph{Multinomial logistic regression model (MLR)}
The multinomial logistic regression analysis included the following set of variables on a nominal level: the three levels of academic achievement scores as outcome variable, and 
 {\it gender}, primary school class level ({\it grade}), and  teacher reports on the nine inattention items {\it SNAP1},$\ldots$,{\it SNAP9} as predictors. 
Generally, the multinomial logistic regression model relates a set of explanatory variables $x_1, \ldots, x_p$ to a set of log-odds, $\log(\pi_2/\pi_1), \ldots \log(\pi_J/\pi_1)$ according to
\begin{equation}
\label{eq_MLR}
\log(\pi_j/\pi_1) = \beta_{j0} + \beta_{j1} x_1 + \cdots + \beta_{jp} x_p
\end{equation}
for $j=2,\ldots,J$. Here, $j = 1$ represents the base level category, $\pi_j = P(\mbox{academic achievement level} = j)$, $\pi_j/\pi_{j'}$ denotes the odds of category $j$ relative 
to $j'$ {(i.e. odds ratio, OR), and $\sum_{j=1}^J \pi_j = 1$ (see e.g. \cite{Bilder2015} for details).
In our case, we let the base level category $j=1$ be the {\it low} mean academic achievement, \textcolor{red}{meaning that the {\it low} was compared separately to 
the {\it medium} and {\it high} categories.}
For computations we used  \textcolor{red}{{\tt \small mnlogit( )} from the {\tt \small statsmodels.formula.api}.} \\

\vspace{3mm}

\paragraph{Classification trees (CART)}
The {\it SNAP1},$\ldots$,{\it SNAP9} items were included together with demographics {(\emph{gender} and primary school glass level ({\it grade})) as predictor variables in a CART analysis \cite{Breiman2001} used to predict level of academic achievement score $\{$\emph{low}, \emph{medium}, \emph{high}$\}$.
In brief, the \emph{root} of the classification tree is the top node and input patterns are passed down the tree such that decisions are made at each node until a terminal 
node (a \emph{leaf}) is reached. At each non-terminal node a question is posed on which a binary split is made such that the ``child" nodes are on average ``purer" than their ``parent". 
A measure of ``impurity" is 
low (i.e. close to $0$) if the probability of the occurrence of a class at a given node for all subsets of patterns reaching that node is concentrated on that class. 
The ``impurity" is maximal \textcolor{red}{(i.e. close to $1$)} if the class probabilities at that node is uniform. 
         
In our analysis we used \textcolor{red}{the {\tt \small DecisionTreeClassifier()}  from {\tt \small sklearn.tree} with impurity {\it criterion=`gini'} and {\it max\_depth=2}
for growing the classification tree (Fig. 2).}

\paragraph{Random forest ensemble learning (RF)}
\textcolor{blue}{Random forest (RF) is an ensemble learning method for classification that constructs a multitude of decision trees at training time and output the mode class among 
the generated classes. The RF algorithm involves the construction of $n$ trees and ensures that each tree uses a different set of data (bootstrapping) 
and a different set of variables (``feature bagging") at each candidate split. 
Thus, RF is less prone to overfitting compared to CART, and will therefore produce more generalisable results \cite{Breiman2001}.  
Moreover, the order of decisions in the hierarchies of trees will reflect the importance of the corresponding feature variables being involved. In our setting,
the variables {\it gender}, {\it grade}, and the {\it SNAP1},$\ldots$,{\it SNAP9} items were included as predictors of the outcome level of academic achievement:
{\it low} ({\it L}), {\it medium} ({\it M}), or {\it high} ({\it H}).
In the analysis we used the {\tt \small RandomForestClassifier()}  from {\tt \small sklearn.ensemble} with impurity {\it criterion=`gini'} ,
{\it n\_estimators=10000}, {\it bootstrap=True}, {\it max\_features=None}, and {\it max\_depth=None}. 
After fitting the forest with the $2491 \times 11$ predictor matrix $X$ and academic achievement outcome $y \in \{L, M, H\} $, i.e. {\tt \small forest.fit(X,y)}, the 
Scikit-learn RF algorithm enables the calculation of {\tt \small forest.feature\_importances\_} which ranking is depticed graphically in Fig.~3.}



\subsubsection*{(c) Prediction using $k$-fold cross-validation}

\textcolor{blue}{From the feature importance step, the top ranked predictors of academic achievement scores were 
selected for a comprehensive classification study using $k$-fold cross-validation to assess prediction properties (accuracy, precision, and recall). 
In this procedure we used both linear classifiers (MLR) and non-linear classifiers (MLP, XGB, SVM, KNN ....). }

\textcolor{blue}{For the $k$-fold cross-validation we used  {\tt StratifiedKFold()}  from {\tt sklearn.model\_selection}  with {\it n\_splits =10} and {\it shuffle=True}. 
For a given fold (split) $1,\ldots,k=10$, fixed pairs of ({\it X\_train, y\_train)} and ({\it X\_test, y\_test)} datasets were provided for each of the six classifiers 
using the {\tt Pipeline} mechanism in Scikit-learn. 
For the individual classifications we used {\tt LogisticRegression()} with {\it solver=`saga'} and {\it multi\_class=`multinomial'}; 
{\tt MLPClassifier()} with ...  {\tt XGBClassifier} from {\bf xgboost} with .... ;  {\tt SVC()} with ....;  {\tt KNeighborsClassifier()} with ...
and the {\tt VotingClassifier()} .
For the assessment we used {\it accuracy\_score}, {\it precision\_score}, {\it recall\_score}, and {\it f1\_score} from {\tt sklearn.metrics} .}


%This procedure randomly splits the dataset into k consecutive folds, of which a 
%single subsample is retained as a validation set, while the remaining k - 1 folds form the training set. The procedure is repeated unit 
%each k subsamples are used once as the validation data. 
%To obtain the best vote for accuracy, we used the following five classifiers: Two linear  classifiers: the multinominal logistic regression 
%( MLR) and the multinominal logistic P.....(MLP), and three nonlinear classifiers: XGB , Support vector machine (SVM), CV  and KNN 
%to vote for overall accuracy, precision and recall. 
%Precision is the number of items correctly labeled as belonging to the positive class divided by the sum of true positives and false positives, 
%also called positive predictive value. Recall is the number of true positives divided by the sum of true positives and false negatives, also known as sensitivity. 
%In the present study we used number of splits = 3, random size = 42, score average = weighted.}

\vspace{5mm}


\section*{Results}

\textcolor{red}{We first report the results from the analysis of feature importance, then the prediction results from $k$-fold cross validation using the six different classifiers.}

\subsection*{Assessment of feature importance} 


\paragraph{Multinomial logistic regression model (MLR)}

%The chi-square test, indicating how much new variance is explained by the baseline 0-model, is statistically significant (255.7, $p < .001$). 

Performing MLR on the complete dataset, {\it gender} significantly predicted whether a child obtained a {\it low} rather than a {\it high} academic achievement score in high school (OR = 0 .60, $p < .002$) as well as a {\it low} rather than a {\it medium} score (OR = $0.79$, $p < 0.001$). This shows that the boys ($1$) were overall more likely to obtain a {\it low} academic achievement score in high school than the girls ($0$). \textcolor{red}{Table 3}. 

Two of the teacher reported inattention items significantly predicted a {\it low} rather than a {\it medium} academic achievement score. The strongest effect was found for an item reflecting problems related to sustained attention, {\it SNAP2} ($p = 0.001$). An odds ratio of $.54$ tells us that for each unit change in the score given by the teacher, the child was almost two times less likely to obtain a {\it medium} compared to a {\it low} academic achievement score ($1/.54 = 1.9$). The second item reflects distractibility, {\it SNAP8} ($p = 0.02$, OR = 0.75), leaving the child with a somewhat increased odds ($1.3$) of obtaining a {\it low} score.  


Predictions from the two inattention items were even stronger when comparing {\it low} to {\it high} academic achievement scores, with the highest estimate on {\it SNAP2} ($p < 0.001$) 
followed by {\it SNAP8} ($p < 0.001$).  The odds ratios show that the child was $2.5$ times more likely to obtain a {\it low} than {\it high} score in high school for each more severe step in problems reported on {\it SNAP2} (OR = $0.40$) and  $1.8$ times more likely for each step on {\it SNAP8} (OR = $0.55$).  
The prediction of {\it low} rather than {\it high} academic achievement score was also significant for two other items reflecting problems related to sustained attention, {\it SNAP1} ($p = 0.05$) and {\it SNAP6} ($p = 0.001$). { With ORs of $0.61$ and $0.48$, the increase was around twofold ($1.6$ and $2,1$, respectively). \it SNAP5} ($p = 0.009$) gave a more surprising result, with a higher likelihood to obtain a high academic achievement level if reported with disorganised behaviour by your primary school teacher.\\

To sum up the results from the MLR, inattentive behaviour associated with problems related to sustained attention and distractibility predicted {\it low} rather than {\it medium}
or {\it high} academic achievement levels in high school, with an overall higher odds-ratio in boys than in girls (\textcolor{red}{Table 3}). \\



  

\vspace{5mm}

% latex table generated in R 3.4.0 by xtable 1.8-2 package
% Mon Jun 12 01:56:24 2017
\begin{table}[H]
\centering
\caption{\bf Multinomial logistic regression model.} 
\begin{tabular}{|llrrrr|rrr|}
  \hline
 %& Estimate & Std..Error & t.value & Pr...t.. & X2.5.. & exp.mlFitD.coefficients. & X97.5.. \\ 
Ref. category: &&&&&&&&\\
Low score & Variable &Estimate & SE & \textbf{z} & \textbf{P$>$$|$z$|$} &  OR &  [0.025, & 0975]\\ 
 \hline
  \hline
% Medium score &&&&&&&\\
Medium score & intercept  & 0.74 & 0.20 & 3.72 & $<$0.001 & 2.09 &   0.35 & 1.12 \\ 
\hline
  & {\it gender} & -0.24 & 0.11 & -2.29 & 0.022 & 0.79 &  -0.44 & -0.03  \\ 
 \hline
  & {\it grade} & -0.15 & 0.07 & -2.51 & 0.023& 0.86 &  0.29 & -0.04  \\ 
 \hline
  & {\it SNAP1} & -0.01 & 0.14 & -0.04 & 0.970 & 0.99 &  -0.27 & 0.26 \\ 
  \hline
  &  {\it SNAP2} & -0.62 & 0.19 & -3.24 & 0.001 & 0.54 &  -1.00 & -0.25  \\ 
  \hline
  &  {\it SNAP3} & 0.06 & 0.19 & 0.31 & 0.757& 1.06 &  -0.32 & 0.44 \\ 
  \hline
  & {\it SNAP4} & 0.21 & 0.23 & 0.89 & 0.374 & 1.23 &  -0.25 & 0.67  \\ 
  \hline
  &  {\it SNAP}5 & 0.01 & 0.22 & 0.04 & 0.970 & 1.01 & -0.42 & 0.44 \\ 
  \hline
  &  {\it SNAP6} & -0.19 & 0.20 & -0.94 & 0.350 &  0.83 &  -0.58 & 0.21  \\ 
   \hline
  &  {\it SNAP7} & 0.16 & 0.26 & 0.60 & 0.550 & 1.17 &  -0.35 & 0.67  \\ 
  \hline
  &  {\it SNAP8} & -0.28 & 0.13 & -2.27 & 0.023 & 0.75 &  -0.53 & -0.04  \\ 
  \hline
  &  {\it SNAP9} & -0.09 & 0.16 & -0.55 & 0.581 & 0.92 &  -0.40 & 0.23  \\ 
   \hline
   \hline
%   High Score &&&&&&&\\
   High score & intercept & 0.78 & 0.20 & 3,88 & $<$0.001 & 2.19 &  0.39 & 1.18  \\ 
  \hline
  &  {\it gender} & -0.51 & 0.11 & -4.83 & $<$0.001 & 0.60 &  -0.72 & -0.30  \\ 
 \hline
 &  {\it grade} & -0.07 & 0.07 & -1.05 & 0.292 & 0.93 &  -0.20 & 0.06  \\ 
 \hline
  &  {\it SNAP1} & -0.50 & 0.18 & -2.80 & 0.005 & 0.61 &  -0.85 & -0.15  \\ 
  \hline
  &  {\it SNAP2} & -0.93 & 0.26 & -3.55 & $<$0.001 & 0.40 &  -1.44 & -0.42  \\ 
  \hline
  &  {\it SNAP3} & -0.09 & 0.25 & -0.38 & 0.704 & 0.91 &  -0.57 & 0.39  \\ 
  \hline
  &  {\it SNAP4} & 0.18 & 0.32 & 0.57& 0.567 & 1.20 &  0.44 & 0.80 \\ 
  \hline
  &  {\it SNAP5} & 0.46 & 0.27 & 1.69 & 0.091 & 1.58 &  -0.07 & 0.99  \\ 
  \hline 
  &  {\it SNAP6} & -0.74 & 0.29 & -2.58 & 0.010 & 0.48 &  -1.30 & -0.18  \\ 
   \hline
  &  {\it SNAP7} & -0.28 & 0.43 & -0.66 & 0.511 & 0.75 &   1.13 & 0.56  \\ 
  \hline
  &  {\it SNAP8} & -0.61 & 0.15 & -4.16 & $<$0.001 & 0.55 &  -0.89 & -0.32  \\ 
  \hline
  &  {\it SNAP9} & -0.14 & 0.19 & -0.74 & 0.462 & 0.87 &  -0.51 & 0.23  \\ 
   \hline

\end{tabular}
%\end{table}
%\vspace{-4mm}
\begin{center}
Reference group = low academic achievment. OR = Odds ratio.\\ 
\end{center}
\end{table}
\vspace{5mm}


\paragraph{Classification trees (CART)}

%\subsubsection*{Tree classification and feature importance}
 The CART analysis, \textcolor{red}{using  maximum depth=$2$},  generated four terminal nodes (Fig.~2). 
 
\vspace{3mm}
 
\begin{center}
---------------------------------\\
 
\textcolor{blue}{Figure 2 about here} \\
 
---------------------------------\\
\end{center}
  
\vspace{3mm}
  
 The first and most important split \textcolor{red}{(the top node \#0)} was on {\it SNAP2}, 
 assessing problems related to sustained attention. The \textcolor{red}{``False" branch at this node, i.e. teachers reporting ``somewhat true" or ``certainly true" on this item, 
 arriving at node \#4  ($11.5$\% of the sample),} were 
 mainly associated with a {\it low} academic achievement score in high school. 
 \textcolor{red}{In this subsample, primary school class level ({\it grade})} did matter. 
 A higher portion of those those with ``somewhat true" or ``certainly true" 
 reports on {\it SNAP2} in the 3$^{\text{rd}}$ and 4$^{\text{th}}$ grades  \textcolor{red}{(node \#6) obtained {\it lower} academic scores than those in 
 the 2$^{\text{nd}}$ grade (node \#5), 71\% and 57\%, respectively}. 

%  (node \#3). A total of 12 \% of the sample was allocated to this node, where only 9 \% obtained the highest and 65 \% the lowest achievement level. If teachers reported no problems on SNAP 2, there was a second split on SNAP 8, assessing distractibility. A teacher report of \emph{somewhat true} or \emph{certainly true}  on this item (14\% of the sample) allocated 41 \% of the children to the lowest academic achievement level (node \#6). 
 
If primary school teachers reported ``not true" on {\it SNAP2},  \textcolor{red}{then the reports on problems related to distractibility ({\it SNAP8}) was important for prediction, 
i.e. node \#1 comprising 88.5\% of the sample.} 
Reporting ``somewhat true" or ``certainly true"  on {\it SNAP8}, i.e. the ``false" branch from node \#1 to node \#3, led to the highest percentages towards a {\it low} 
academic achievement score (42\%), while a ``not true" report (node \#2, 74.5\% of the sample) was associated with the highest percentages towards the {\it high} score (39\%). \\
	
To sum up the results from the CART analysis, problems related to sustained attention ({\it SNAP2}) or distractibility ({\it SNAP8}) were important predictors of {\it low} school 
academic achievement scores in  25.5\% (nodes \#3 and \#4) of the children.  \textcolor{red}{Primary school class {\it grade} did also matter, with a higher percentage of children 
obtaining a {\it low} score when assessments were done at 3$^{\text{nd}}$ or 4$^{\text{th}}$ grade.}  \\ 

\paragraph{Random forest ensemble learning (RF)}
\textcolor{red}{The random forest algorithm with $10000$ trees further explored feature importances in the cohort. 
Figure~3 shows the ranked importance of the $11$ predictor variables, 
confirming the main findings from the MLR and CART analyses, where to top three most important features were {\it SNAP2}  $>$ {\it grade}  $>$  {\it SNAP8}.}  \\

\vspace{3mm}
 
\begin{center}
---------------------------------\\
 
\textcolor{blue}{Figure 3 about here} \\
 
---------------------------------\\
\end{center}
  
\vspace{3mm}
  

\subsection*{Prediction using $k$-fold cross-validation}

The cross-validation procedure was performed separately for the three top features selected by the RF analysis. Gender was included due to its effect upon both the SNAP-IV items and the academic achievement score, and the statistically significant effect revealed by the MLR.  
Table 4 shows the results from the selected classifiers and the overall voting on the accuracy, precision and recall measures. All values were above chance level for the three categories of academic achievement scores (i.e., $>$ 33\%). 
%Although these results confirm the importance of the selected variables, they do also show that a high percentage was left unexplained.  %



\begin{table}[!ht]
\begin{adjustwidth}{-7mm}{0in} % Comment out/remove adjustwidth environment if table fits in text column.
\centering
%\begin{small}
\caption{\bf Classification performance using $k$-fold cross-validation ($k=10$).} 
\vspace{5mm}
\begin{tabular}{|lrrrrrr|}
\hline
 & MLR & MLP & XGB & SVM & KNN & Voting\\\hline
 \emph{Accuracy} &0.43 (0.03) & 0.43 (0.03) &0.44 (0.02) &0.44 (0.02) & 0.39 (0.03) & 0.42 (0.02)\\
 \emph{Precision} & 0.43 (0.04) & 0.42 (0.04) & 0.45 (0.03) & 0.45 (0.04) & 0.38 (0.03) & 0.42 (0.03)\\
 \emph{Recall} & 0.43 (0.03) & 0.43 (0.03) & 0.44 (0.02) & 0.44 (0.02) & 0.39 (0.03) & 0.42 (0.02) \\
 \hline
\end{tabular} 
\label{crossvalidation}
\end{adjustwidth}
\vspace{2mm}

\textit{Note:} MLR = multinomial logistic regression; MLP = multilayer perceptron; XGB = extreme gradient boosting; SVN = support vector machine; 
KNN = k-nearest neighbour; Voting = voting classifier across MLR, MLP, XGB, SVN, KNN. \\ 

%\end{small}
%\end{adjustwidth}
\end{table}

\section*{Discussion}
\subsection*{Summary of results}
The present study asked if specific features of inattentive behaviour in primary school - as reported by teachers - act as predictors of academic achievement in high school. Different types of multivariate analyses were used to handle the set of categorical variables.  \textcolor{red}{Overall, items reflecting problems related to sustained attention and distractibility were selected as the two most important features of inattention in predicting the achievement score. Gender and a proxy for age (primary school class level) were added as important features by the MLR analysis. The CART analysis showed that as many as 25.5\% of the children were reported with either of the two inattention problems, and that these children had a high risk of obtaining a low academic achievement score. Age  when assessed by their primary school teachers was of some importance, in that the chance of obtaining a low achievement score was somewhat lower when reported with problems in the  2$^{\text{nd}}$ than in higher grades (3$^{\text{rd}}$ and 4$^{\text{th}}$  grades). Age and the items reflecting sustained attention and distractibility were also identified with the highest importance by the RF analysis, suggesting that these results can be generalised to other samples. This was confirmed by the cross-validation analyses.}\\

% 
%Taken together, the results suggest that children reported with problems related to sustained attention and distractibility, were allocated to terminal nodes characterised by a predominance of low academic achievement scores in high school. \\

\subsection*{Early predictors of academic achievement in high school}
The present results showed that problems related to sustained attention and distractibility in primary school are important drivers of poor academic performance in high school. By this, the results partly overlapped with findings previously reported in a study by Holmberg et al. \cite{Holmberg2014}, where  teacher reports of failure to finish a task were found to be one of the main factors explaining academic outcome. 
Our study add to this by revealing the importance of problems related to distractibility. The MLR analysis showed that this problem was associated with an almost two-fold increase in OR of an achievement score in the lower than higher end of the scale. Its importance as a predictor of poor achievement scores was also supported by the CART analysis, \textcolor{red}{with the strongest effect when reported as a problem in the  3$^{\text{rd}}$ and 4$^{\text{th}}$ grades.}  In a class situation, the relation between the two is obvious. A child with the ability to stay focused on a task over a longer period of time is expected to be less disturbed by habits and cues in the environment than a child with poor vigilance.  This enables the child to obtain the basic skills and knowledge that are of importance to the academic achievement scores as the curriculum become more complex at higher grade levels. \\

\textcolor{red}{Inclusion of information about nine aspects of inattentive behaviour separately in the statistical analyses revealed their relative importance to high school academic performance.} Most previous studies have defined inattention as a sum-score from reports of problems reflecting a range of different behaviours. A significant relation between such a sum score and academic achievement was shown in one of our previous studies, including subsamples from BCS and the Berkeley Girls with ADHD Longitudinal Study (BGALS). Inattention was found to be significant across these two culturally and diagnostically diverse groups, and the effect was over and above the effect of demographics and intellectual function \cite{Lundervold2017}. The present results indicate that the effect on academic achievement is driven by a few features defined within the full inattention score.  \\

The cross-diagnostically effect of inattention was confirmed by the present study. Although  inattention is one of the core symptoms of ADHD,  the importance of inattentive behaviour in explaining future academic success is definitely not restricted to a diagnostic category; the present study documented this effect in a population-based sample.  \textcolor{red}{However, although a high proportion of children obtaining a low academic achievement score were reported as inattentive by their primary school teachers, the cross-validation analyses revealed that more information about the child is needed to obtain an improved validation of the prediction. This probably reflects both the instability of inattentive behaviour and the large number of co-existing and new challenges influencing a child through childhood and adolescence.  Previous studies have for example shown the  importance of socio-economic factors in general (e.g., \cite{Russel2015}), with some cultural differences regarding the importance of its subcomponents \cite{Boe2012} and consequences \cite{Ellertsen2016}. Further studies should thus include a larger number of predictors and a more diverse sample than in the present study. }
%Identification of these factors awaits further longitudinal studies. \\

Taken together, the present results should inspire assessment and treatment efforts in primary school children vulnerable to distractibility and with problems to sustain their attention in school-related work. 
The close relationship between inattentive behaviour and cognitive function \cite{Berger2013, Berger2015} has lead to increased popularity of presenting cognitive training programs to school children with ADHD (see e.g., \cite{Rapport2013, Tamm2017}). A sole focus on cognitive training of the child is, however, not expected to lead to successful alleviation of the inattentive behaviour described in the present paper. This was supported by the results from the meta-analysis presented by Cortese and collaborators  \cite{Cortese2015}, showing that cognitive training procedures had limited effects on ADHD symptoms. Positive contributions from parents and teachers seems to be essential (see e.g., \cite{Pfiffner2014}). Whereas parent-focused training produces improvements in negative parenting and impairment at home, incorporation of child skill training and teacher consultation may be necessary to produce improvements at school \cite{Haack2016}. 

Gender turned out to be another important predictor. Girls were reported by their primary school teachers to have less inattention symptoms and to obtain higher academic achievement in high school than boys.  Although gender was identified as one of the main predictors of academic achievement scores in the feature extraction by the MLR analysis, it was not selected among the top features of importance by the the CART and RF analyses.  Further gender balanced longitudinal studies of functional outcomes of early inattentive behaviour are warranted. \\

%
%\subsection*{Selection of statistical methods}
%Questionnaire data with only a few response categories are commonly used in psychological research. The  relevance of the present analytic approach is therefore not restricted to the topic of the present study. The two statistical methods in the present study were selected to handle situations where predictor variables with a few number of categories may hide complex interactions. This approach is therefore clearly appropriate when investigating predictions from scales representing complex concept like inattention. Here,  the weights of different problem areas and their interactions may give information of importance to understand how to identify essential problems in a child. 
% In the present study we included a multinominal logit model to handle the categorical characteristics of the data. By the multinominal logit model, the generated coefficients were easily converted to probability values. In the setting of this study, the coefficient indicated the probability that any child will be classified in one of the academic achievement categories given the reports of inattention, age and gender. This method is, however, not appropriate to reveal behaviour patterns that may be hidden by complex interactions between the teacher reports on the nine  inattention items. We therefore included CART, a method appropriate with our aim to investigate if specific problem areas are of importance to predict future academic outcome. 
% 
%

 \subsection*{Strengths and Limitations}
The large population-based sample of high school students followed from childhood, inclusion of a standardised questionnaire assessing inattention, and inclusion of academic achievement scores from official National registers are main strengths of the present study. Another strength is the inclusion of statistical methods - \textcolor{red}{the use of the MLR , CART, RF algorithms for feature selection, and the comprehensive cross-validation procedure.} We believe that the relevance of the present analytic approach is clearly not restricted to the topic of the present study, in that questionnaire data with only a few response categories are commonly used in psychological research. \\

%I Although the statistical analyses have several strengths, applications of recursive partitioning methods in psychology also reveal common misperceptions and pitfalls. For example, Luellen et al. (2005) suspected that the CART methods could overfit (i.e., adapt too closely to random variations in the learning sample). This underscores the need to confirm results across different methods or inclusion of statistical validation procedures.  
%
%By adding recursive partitioning, a high number of threes were generated. This is important, in that the selection of a splitting variable will strongly depend on the distribution in the learning sample. By repeating the procedure as in the present study, the prediction is expected to be more correct than when based on a single tree. 

In spite of the strengths and the importance of the present study, several limitations must be mentioned. Inclusion of very few features when predicting an outcome about 10 years ahead, is an obvious limitation of the present study. A stronger model could have been obtained by including results from psychometric test assessing vigilance and distractibility, similar to the one developed by Cassuto et al. \cite{Cassuto2013}, or more ecological valid virtual reality test as the one described by Pelham et al. \cite{Pelham2011}. Inclusion of teacher reports only may also be considered as a limitation.  Furthermore, stronger conclusions could have been obtained by including information from repeated inattention reports to understand the trajectory from early symptoms of inattention to function in adolescence and adulthood. The importance of the latter was demonstrated in a study by Pingault and collaborators \cite{Pingault2014}, showing that increase in symptoms of inattention during childhood really matters when it comes to school graduation failure. Such studies are important and should include analysis of behavioural patterns, because a specific pattern of vigilance and distraction was suggested by the present study. \textcolor{red}{Finally, academic achievement level did not reflect overall high school achievement, in that it was operationalised as the mean of grades for one semester only.}\\


%\paragraph{Figure 1} The data organisation for the sample. 
%{\it gender} is $0$ (girls) or $1$ (boys), {\it grade} is primary school class level, $2$, $3$, or $4$. The  three-level Likert items from Snap-IV: SNAP1$, \ldots, $ SNAP9 each have three possible values: $0$ = \emph{not true}; $1$ = \emph{somewhat true}; $2$ = \emph{certainly true}) 

\vspace{5mm}
\section*{Figure legends}
\paragraph{Figure 1} The predictor data (explanatory variables), the academic achievement outcome, and the types of classification analyses being performed. 
Data values are represented as grey level heat maps. See 
Fig. 1 for explanation of variables.\,
MLR = Multinomial logistic regression, CART = Classification and regression trees. \\



\vspace{5mm}

\paragraph{Figure 2} Fitted classification tree (CART analysis),  including the predictor variables SNAP-IV items $1$ to $9$ ($0$ = \emph{not true}; $1$ = \emph{somewhat true};
 $2$ = \emph{certainly true}); gender (0 = \emph{girl}; 1 = \emph{boy}); grade (primarty school class level 2, 3, 4) and the academic achievement outcome (H = \emph{high}, L = \emph{low},  M = \emph{medium}, where the three occurrence frequencies in the node boxes are given in alphabetical order). The percentage in each node box denote the
 percentage of samples routed to that particular node - where the root node will contain 100\% of the samples, and a leaf node will contain the least number of samples along 
 a rooted path in the decision tree.
The node numbers are given on top of each node box. For each split decision, \emph{`yes'} denotes that the corresponding statement is \emph{ false} 
and then pointing to the left child node
(that is either a new internal decision node or a final leaf node), and \emph{`no'}
denotes that the corresponding statement is \emph{true} and then pointing to the right child node (that is either a new internal decision node or a final leaf node).  

%The tree is plotted using the {\small \tt fancyRpartPlot()} function from Graham Williams {\bf rattle} package
%({\small \url{http://rattle.togaware.com}})


\section*{Acknowledgement}
The present study was supported by the Centre for Child and Adolescent Mental Health and Welfare, Uni health, Uni
Research, Bergen, Norway, and was also funded by the University of Bergen, the Norwegian Directorate for Health and
Social Affairs, and the Western Norway Regional Health Authority. We are grateful to the children, parents and teachers
participating in the Bergen Child Study (BCS) and members of the BCS project group for making the study possible. A
special thank to Professor Christopher Gillberg, who initiated this study.

\newpage

%
%% Include only the SI item label in the paragraph heading. Use the \nameref{label} command to cite SI items in the text.
%\paragraph*{S1 Fig.}
%\label{S1_Fig}
%{\bf Bold the title sentence.} Add descriptive text after the title of the item (optional).
%
%\paragraph*{S2 Fig.}
%\label{S2_Fig}
%{\bf Lorem ipsum.} Maecenas convallis mauris sit amet sem ultrices gravida. Etiam eget sapien nibh. Sed ac ipsum eget enim egestas ullamcorper nec euismod ligula. Curabitur fringilla pulvinar lectus consectetur pellentesque.
%
%\paragraph*{S1 File.}
%\label{S1_File}
%{\bf Lorem ipsum.}  Maecenas convallis mauris sit amet sem ultrices gravida. Etiam eget sapien nibh. Sed ac ipsum eget enim egestas ullamcorper nec euismod ligula. Curabitur fringilla pulvinar lectus consectetur pellentesque.
%
%\paragraph*{S1 Video.}
%\label{S1_Video}
%{\bf Lorem ipsum.}  Maecenas convallis mauris sit amet sem ultrices gravida. Etiam eget sapien nibh. Sed ac ipsum eget enim egestas ullamcorper nec euismod ligula. Curabitur fringilla pulvinar lectus consectetur pellentesque.
%
%\paragraph*{S1 Appendix.}
%\label{S1_Appendix}
%{\bf Lorem ipsum.} Maecenas convallis mauris sit amet sem ultrices gravida. Etiam eget sapien nibh. Sed ac ipsum eget enim egestas ullamcorper nec euismod ligula. Curabitur fringilla pulvinar lectus consectetur pellentesque.
%
%\paragraph*{S1 Table.}
%\label{S1_Table}
%{\bf Lorem ipsum.} Maecenas convallis mauris sit amet sem ultrices gravida. Etiam eget sapien nibh. Sed ac ipsum eget enim egestas ullamcorper nec euismod ligula. Curabitur fringilla pulvinar lectus consectetur pellentesque.
%
%\section*{Acknowledgments}
%Cras egestas velit mauris, eu mollis turpis pellentesque sit amet. Interdum et malesuada fames ac ante ipsum primis in faucibus. Nam id pretium nisi. Sed ac quam id nisi malesuada congue. Sed interdum aliquet augue, at pellentesque quam rhoncus vitae.
%
%\nolinenumbers
%
% Either type in your references using
% \begin{thebibliography}{}
% \bibitem{}
% Text
% \end{thebibliography}
%
% or
%
% Compile your BiBTeX database using our plos2015.bst
% style file and paste the contents of your .bbl file
% here. See http://journals.plos.org/plosone/s/latex for 
% step-by-step instructions.
% 

\begin{thebibliography}{10}

\bibitem {APA2013}
American Psychiatric Association. 
\newblock Diagnostic and Statistical Manual of Mental
Disorders (5th ed.). 
\newblock Washington, DC: Author; 2013.

\bibitem{Becker2013}
Becker SP, Luebbe AM, Fite PJ, Greening L,
  Stoppelbein L.
\newblock Oppositional defiant disorder symptoms in relation to psychopathic
  traits and aggression among psychiatrically hospitalized children: {ADHD}
  symptoms as a potential moderator.
\newblock Aggressive Behavior. 2013;39(3):201-211. doi: 10.1002/ab.21471

\bibitem{Bellanti2000}
Bellanti CJ, Bierman, KL.
\newblock Disentangling the impact of low cognitive ability and inattention on
  social behavior and peer relationships. Conduct problems prevention research
  group.
\newblock Journal of Clinical Child Psychology. 2000;29(1), 66-75.

\bibitem{Berger2015}
Berger I, Remington A, Leitner Y, Leviton A.
\newblock Brain development and the attention spectrum.
\newblock Frontiers in Human Neuroscience. 2015;9:23. doi: 10.1038/tp.2017.164

\bibitem{Berger2013}
Berger I, Slobodin O, Aboud M, Melamed J, Cassuto H.
\newblock Maturational delay in {ADHD}: evidence from {CPT}.
\newblock Frontiers in Human Neuroscience. 2013;7:691. doi: 10.3389/fnhum.2013.00691

\bibitem{Berry2014}
Berry AS, Demeter E, Sabhapathy S, English BA,
  Blakely RD, Sarter M, Lustig C.
\newblock Disposed to distraction: genetic variation in the cholinergic system
  influences distractibility but not time-on-task effects.
\newblock Journal of Cognitive Neuroscience. 2014;26(9):1981-1991.  doi: 10.1162/jocn\_a\_00607.

%\bibitem{Biederman2004}
%Joseph Biederman, Stephen~V. Faraone, Michael~C. Monuteaux, Marie Bober, and
%  Elizabeth Cadogen.
%\newblock Gender effects on {A}ttention-{D}eficit {H}yperactivity disorder in
%  adults, revisited.
%\newblock {\em Biol Psychiatry}, 55(7):692--700, Apr 2004.

\bibitem{Bilder2015}
Bilder CR,  Loughin TM.
\newblock Analysis of Categorical Data with R.
\newblock CRC Press; 2015.

\bibitem{Breiman2001}
Breiman L.
\newblock Random Forests. 
\newblock Machine Learning. 2001;45(1):5-32. doi:10.1023/A:1010933404324.

\bibitem{Bussing2003c}
Bussing R, Zima BT, Gary FA, Garvan CW.
\newblock Barriers to detection, help-seeking, and service use for children
  with {ADHD} symptoms.
\newblock The journal of Behavioral Health Services \& Research. 2003;30:176-189.

\bibitem{Boe2012}
B\o{}e T, \O{}verland S, Lundervold AJ, Hysing M.
\newblock Socioeconomic status and children's mental health: results from the Bergen Child Study.
\newblock Soc Psychiatry Psychiatr Epidemiol. 2012;47(10):1557-66. doi: 10.1007/s00127-011-0462-9. 

\bibitem{Capriotti2017}
Capriotti MR, Pfiffner LJ.
\newblock Patterns and predictors of service utilization among youth with
  {ADHD} - predominantly inattentive presentation.
\newblock Journal of Attention Disorders. 2017;1:1087054716677817. doi: 10.1177/1087054716677817.

\bibitem{Cassuto2013}
Cassuto H, Ben-Simon A, Berger I.
\newblock Using environmental distractors in the diagnosis of {ADHD}.
\newblock Frontiers in Human Neuroscience. 2013;7:805. doi: 10.3389/fnhum.2013.00805.

\bibitem{Connors2012}
Connors LL, Connolly J, Toplak ME.
\newblock Self-reported inattention in early adolescence in a community sample.
\newblock Journal of Attention Disorders.  2012;16(1):60-70. doi: 10.1177/1087054710379734.

\bibitem{Cortese2015}
Cortese S, Ferrin M, Brandeis D, Buitelaar J, Daley D,
  Dittmann RW, etal.
\newblock Cognitive training for attention-deficit/hyperactivity disorder:
  meta-analysis of clinical and neuropsychological outcomes from randomized
  controlled trials.
\newblock Journal of the American Academy of Child and Adolescent
  Psychiatry. 2015;54:164-174. doi: 10.1016/j.jaac.2014.12.010.
  
\bibitem{Ellertsen2016}
Ellertsen T, Thorsen AL, HolmSE, B\o{}e T, S\o{}rensen L, Lundervold,
AJ.
\newblock A weak association between socioeconomic status and cognition
in a sample of Norwegian children. 
\newblock Scandinavian Journal of Psychology, 2016;57: 399?405. doi: 10.1111/
sjop.12324
  
 \bibitem{Fried2016}
 Fried R, Petty C, Faraone SV, Hyder LL, Day H, Biederman J.
\newblock Is ADHD a risk factor for high school dropout? A controlled
study. 
 \newblock Journal of Attention Disorder. 2016;20, 383-389. doi: 10.1177/1087054712473180.
 
 
\bibitem{Garner2013}
Garner AA, O'connor, BC, Narad ME, Tamm L, Simon J,
 Epstein JN.
\newblock The relationship between {ADHD} symptom dimensions, clinical
  correlates, and functional impairments.
\newblock Journal of Developmental and Behavioral Pediatrics. 2013;34(7):469-477. doi: 10.1097/DBP.0b013e3182a39890.

%\bibitem{Graetz2006}
%Brian~W. Graetz, Michael~G. Sawyer, Peter Baghurst, and Kerry Ettridge.
%\newblock Are {ADHD} gender patterns moderated by sample source?
%\newblock {\em Journal of Attention Disorders}, 10(1):36--43, Aug 2006.

\bibitem{Gray2014}
Gray SAO, Carter AS, Briggs-Gowan MJ, Jones SM,
Wagmiller RL.
\newblock Growth trajectories of early aggression, overactivity, and
  inattention: Relations to second-grade reading.
\newblock Developmental Psychology. 2014;50(9):2255-2263. doi: 10.1037/a0037367. 

\bibitem{Haack2016}
Haack LM, Villodas M, McBurnett K, Hinshaw S, Pfiffner LJ.
\newblock Parenting as a mechanism of change in psychosocial treatment for
  youth with {ADHD}, predominantly inattentive presentation.
\newblock Journal of Abnormal Child Psychology. 2016;45(5):841-855. doi: 10.1007/s10802-016-0199-8.

%\bibitem{Halmoey2009}
%Anne Halm\o{}y, Ole~Bernt Fasmer, Christopher Gillberg, and Jan Haavik.
%\newblock Occupational outcome in adult {ADHD}: impact of symptom profile,
%  comorbid psychiatric problems, and treatment: a cross-sectional study of 414
%  clinically diagnosed adult {ADHD} patients.
%\newblock {\em Journal of Attention Disorders}, 13(2):175--187, Sep 2009.

\bibitem{Heiervang2007}
Heiervang E, Stormark KM, Lundervold AJ, Heimann M, Goodman R, Posserud M, etal.  
\newblock Psychiatric disorders in {N}orwegian 8- to 10-year-olds: an
  epidemiological survey of prevalence, risk factors, and service use.
\newblock Journal of the American Academy of Child and Adolescent
  Psychiatry. 2007;46(4):438-447. doi: 10.1097/chi.0b013e31803062bf

\bibitem{Holmberg2014}
Holmberg K, {B\"{o}lte} S.
\newblock Do symptoms of {ADHD} at ages 7 and 10 predict academic outcome at
  age 16 in the general population?
\newblock Journal of Attention Disorders. 2014;18(8):635--645. doi: 10.1177/1087054712452136.

\bibitem{Lee2006}
Lee SS, Hinshaw SP.
\newblock Predictors of adolescent functioning in girls with attention deficit
  hyperactivity disorder ({ADHD}): the role of childhood {ADHD}, conduct
  problems, and peer status.
\newblock Journal of Clinical Child and Adolescent Psychology. 2006;35(3), 356-368. doi: 10.1207/s15374424jccp3503\_2

\bibitem{Lundervold2011}
Lundervold AJ, Adolfsdottir S., Halleland H., Halm\o{}y A, 
  Plessen K, Haavik J.
\newblock Attention network test in adults with ADHD -- the impact of affective
  fluctuations.
\newblock Behavior and Brain Function. 2011;27;7:27.  doi: 10.1186/1744-9081-7-27.

\bibitem{Lundervold2017}
Lundervold  AJ, Meza J, Hinshaw SP.
\newblock Parent rated symptoms of inattention in childhood predict high school academic achievement across two culturally and diagnostically diverse samples.
\newblock Frontiers of Psychology. 2017:8,1436. doi: 10.3389/fpsyg.2017.01436.

\bibitem{Metcalfe2013}
Metcalfe LA, Harvey EA, and Laws HB.
\newblock The longitudinal relation between academic/cognitive skills and
  externalizing behavior problems in preschool children.
\newblock Journal of Educational Psychology. 2013;105:881-894. doi: 10.1037/a0032624

\bibitem{Ohan2009}
Ohan JL, Visser TAW.
\newblock Why is there a gender gap in children presenting for attention
  deficit/hyperactivity disorder services?
\newblock Journal of Clinical Child and Adolescent Psychology. 2009;38(5):650-60. doi: 10.1080/15374410903103627.
 

\bibitem{Owens2017}
Owens EB, Zalecki C, Gillette P, Hinshaw SP.
\newblock Girls with childhood ADHD as adults: Cross-domain outcomes by
  diagnostic persistence.
\newblock Journal of Consulting and Clinical Psychology. 2017;85(7):723-736. doi: 10.1037/ccp0000217.

\bibitem{Pelham2011}
Pelham WE, Waschbusch DA, Hoza B, Gnagy EM,
  Greiner AR, Sams SE, etal. 
  \newblock Music and video as distractors for boys with {ADHD} in the classroom:
  comparison with controls, individual differences, and medication effects.
\newblock Journal of Abnormal Child Psychology. 2011;39(8):1085-1098. doi: 10.1007/s10802-011-9529\-z.

\bibitem{Pfiffner2014}
Pfiffner LJ, Haack LM.
\newblock Behavior management for school-aged children with {ADHD}.
\newblock Child and Adolescent Psychiatric Clinics of North America. 2014;23:731-746. doi: 10.1016/j.chc.2014.05.014.

\bibitem{Pingault2014}
Pingault J, Cote SM, Vitaro F, Falissard B,
Genolini C, Tremblay RE.
\newblock The developmental course of childhood inattention symptoms uniquely
  predicts educational attainment: a 16-year longitudinal study.
\newblock Psychiatry Research. 2014;219(3):707-709. doi: 10.1016/j.psychres.2014.06.022.

\bibitem{Polderman2010}
Polderman TJ, Boomsma DI, Bartels M, Verhulst FC, Huizink AC.
\newblock A systematic review of prospective studies on attention problems and
  academic achievement.
\newblock Acta Psychiatrica Scandinavica. 2010;122(4):271-284. doi: 10.1111/j.1600-0447.2010.01568.x.

\bibitem{Rapport2013}
Rapport MD, Orban SA, Kofler MJ, Friedman LM.
\newblock Do programs designed to train working memory, other executive
  functions, and attention benefit children with {ADHD}? a meta-analytic review
  of cognitive, academic, and behavioral outcomes.
\newblock Clinical Psychology Review. 2013;33(8):1237-1252. doi: 10.1016/j.cpr.2013.08.005.

\bibitem{Rescorla2007}
Rescorla L, Achenbach TM, Ivanova MY, Dumenci L, 
  Almqvist F, Bilenberg N, etal. 
\newblock Epidemiological comparisons of problems and positive qualities
  reported by adolescents in 24 countries.
\newblock Journal of Consulting and Clinical Psychology. 2007;75(2):351-358. doi: 10.1037/0022-006X.75.2.351

\bibitem{Ripley1996}
Ripley, BD.
\newblock Pattern Recognition and Neural Networks.
\newblock Cambridge University Press;1996.

\bibitem{Russel2015}
Russell EA, Ford T, Russell G.
\newblock Socioeconomic Associations with ADHD: Findings from a Mediation Analysis.
\newblock PLoS One. 2015;10(6). doi: 10.1371/journal.pone.0128248. 

\bibitem{Salla2016}
Salla J, Michel G, Pingault JB, Lacourse E, 
  Paquin S, Galera C, etal. 
\newblock Childhood trajectories of inattention-hyperactivity and academic
  achievement at 12 years.
\newblock European Child and Adolescent Psychiatry. 2016;25:1195-1206. doi: 10.1037/0022-006X.75.2.351.

\bibitem{Sivertsen2015}
Sivertsen B, Glozier N, Harvey AG, Hysing, M.
\newblock Academic performance in adolescents with delayed sleep phase.
\newblock Sleep Medicine. 2015;16(9):1084-1090. doi: 10.1016/j.sleep.2015.04.011.

\bibitem{Strobl2009}
Strobl C, Malley J, Tutz G.
\newblock An introduction to recursive partitioning: rationale, application,
  and characteristics of classification and regression trees, bagging, and
  random forests.
\newblock Psychological Methods. 2009;14(4):323-348. doi: 10.1037/a0016973.

\bibitem{Swanson1992}
Swanson JM.
\newblock School-based assessments and interventions for ADD students. 
\newblock KC publishing;1992.

\bibitem{Tamm2017}
Tamm L, Epstein JN, Loren REA, Becker SP, Brenner SB, Bamberger ME, etal.
\newblock Generating attention, inhibition, and memory: A pilot randomized
  trial for preschoolers with executive functioning deficits.
\newblock Journal of Clinical Child and Adolescent Psychology. 2017;20:1-15. doi: 10.1080/15374416.2016.1266645. 

%
%\bibitem{Vildalen2016}
%Victoria~U Vildalen, Erlend~J Brevik, Jan Haavik, and Astri~J Lundervold.
%\newblock Females with adhd report more severe symptoms than males on the adult
%  adhd self-report scale.
%\newblock {\em Journal of Attention Disorders}, July 2016.

\end{thebibliography}

%\begin{thebibliography}{10}
%
%
%\bibitem{bib1}
%Conant GC, Wolfe KH.
%\newblock {{T}urning a hobby into a job: how duplicated genes find new
%  functions}.
%\newblock Nat Rev Genet. 2008 Dec;9(12):938--950.
%
%\bibitem{bib2}
%Ohno S.
%\newblock Evolution by gene duplication.
%\newblock London: George Alien \& Unwin Ltd. Berlin, Heidelberg and New York:
%  Springer-Verlag.; 1970.
%
%\bibitem{bib3}
%Magwire MM, Bayer F, Webster CL, Cao C, Jiggins FM.
%\newblock {{S}uccessive increases in the resistance of {D}rosophila to viral
%  infection through a transposon insertion followed by a {D}uplication}.
%\newblock PLoS Genet. 2011 Oct;7(10):e1002337.
%
%\end{thebibliography}
%
%

\end{document}

